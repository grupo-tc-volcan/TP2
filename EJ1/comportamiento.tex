\subsection{Fen\'omenos que afectan al comportamiento de los circuitos} %VER SI DEJAR PARA C/CASO O AL FINAL, for BOTH, o al ppio introductoriamente
	\subsubsection*{Efecto de slew rate (SR)}
	\subsubsection*{Distorsi\'on de cruce por cero (Cross-Over Distortion)}
	\subsubsection*{Gain Bandwidth Product (GBP)}
	\subsubsection*{saturación}

\subsection{Condiciones de comportamiento lineal del circuito}

\subsubsection{An\'alisis te\'orico}

La tensión de entrada máxima del circuito está limitada principalmente por el slew rate slew rate y la saturaci\'on. 

\subsubsection*{Influenccia del slew rate en $V_{in_{max}}$}

Partiendo de:
\begin{equation}
\begin{cases}
	SR = m\'ax\bigg\{\frac{ dV_{out}}{dt}\bigg\} \\
	V_{in} (\omega, t) = V_{in_{max}} \cdot sin(\omega t) \\
	V_{out} (\omega, t) = G(\omega) \cdot V_{in_{max}} \cdot sin(\omega t)
\end{cases}
\label{srecs}
\end{equation}
 
Siendo $SR$ el slew rate, $V_{in}$ y $V_{out}$ las se\~nales de entrada y de salida respectivamente y $G(\omega)$ la ganancia del circuito.

Dado que

\begin{equation}
	\frac{dV_{out}}{dt} = G(\omega) \cdot V_{in_{max}} \cdot \omega \cdot cos(\omega t)
\label{deriv}
\end{equation}

Maximizando la ecuaci\'on \ref{deriv} se obtiene que:

\begin{equation}
	SR = m\'ax\bigg\{\frac{dV_{out}}{dt}\bigg\} = G(\omega) \cdot V_{in_{max}} 
\label{max}
\end{equation}

Despejando de la ecuaci\'on \ref{max}:

\begin{equation}
	 V_{in_{max}}  = \frac{SR}{G(\omega)}
\end{equation}


\subsubsection*{Influenccia de la saturaci\'on en $V_{in_{max}}$}
