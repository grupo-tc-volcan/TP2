\section{Ejercicio 1: Comportamiento de amplificadores operacionales}
En este ejercicio se analizan distintas caracter\'isticas de dos circuitos
 con amplificadores operacionales. Primero se estudia el circuito de la figura 
 \ref{c1}, cuya configuraci\'on es inversora. Luego se analiza el 
 circuirto \ref{c2} de configuraci\'on no inversora. 

\begin{figure}[h!]
 \begin{center}
    \begin{circuitikz}
\ctikzset{bipoles/length=1cm}
\draw
(0, 0) node[op amp] (opamp) {}
(opamp.-) to[short] (-2,0.35)
to[R,l_=$R_1$,-o] (-4, 0.35) node[anchor=east]{IN}
(-4,-1.5) to[sV,v=$V_{in}$] (-4,0.35)
(-4,-1.5) node[ground]{}

(-2,0.35) to[R=$R_3$] (-2,-1.5) node[ground]{}

(opamp.-) to[short,*-] ++(0,0.5) coordinate (leftC)
to[R=$R_2$] (leftC -| opamp.out)
to[short,-*] (opamp.out)  node[anchor=west]{OUT}

(opamp.+) -- (-1,-0.35) to (-1,-1.5) node[ground]{}
(opamp.out) to[short] (0.85,-0.4)
%to[R=$R_4$] (0.85,-1)  to[short] (0.85,-1.5)node[ground]{}
 to[R=$R_4$] (0.85,-1.5) node[ground]{}
;
    \end{circuitikz}
    \caption{Configuraci\'on inversora}
\label{c1}
\end{center}
\end{figure}

\begin{figure}[h!]
 \begin{center}
    \begin{circuitikz}
\ctikzset{bipoles/length=1cm}
\draw
(0, 0) node[op amp] (opamp) {}
%(opamp.-) to[short] (-2,0.35)
%to[R,l_=$R_1$,-o] (-4, 0.35) 
%(-4,0.35) to[short] (-4,-1.5) node[ground]{}
(opamp.-) ++(0,0.5) coordinate (leftC) to[R,l_=$R_1$] (-3,0.8) node[ground]{}

(opamp.-) to[short,*-] ++(0,0.5) coordinate (leftC)
to[R=$R_2$] (leftC -| opamp.out)
to[short,-*] (opamp.out)  node[anchor=west]{OUT}

(opamp.+)  to[short] (-1,-0.35)
to[R=$R_3$] (-3,-0.35) 
(-1,-0.35) to[R=$R_4$] (-1,-2) node[ground]{}

(-3,-2) to[sV,v=$V_{in}$] (-3,-0.35) node[anchor=east]{IN}
(-3,-2) node[ground]{}
;
    \end{circuitikz}
    \caption{Configuraci\'on no inversora}
\label{c2}
\end{center}
\end{figure}

Para llevar a cabo el estudio
 de estos circuitos, se utilizó el amplificador operacional LM324 con las 
 siguientes características, sacadas de hojas de datos
 \footnote{Hojas de datos del operacional LM324: http://www.ti.com/lit/ds/symlink/lm324-n.pdf}:

 \begin{equation}
	\begin{cases}
		SR = 0,5\frac{V}{\mu s}\\
		A_{V0} = 100000\\
		f_p = 10Hz\\
		GBP = 1MHz
	\end{cases}
	\label{lm324}
\end{equation}

Siendo $SR$ el slew rate, $A_{V0}$ el $A_{vol}$ finito, $f_p$ la frecuencia del polo dominante y $GBP$ el bandwidth product.

A lo largo de esta sección se analiza los siguientes tres casos, tanto para
el circuito \ref{c1} como el \ref{c2}.

\begin{table}[h!]
	\centering
	\begin{tabular}{c c c c c}%
		\bfseries  & $\bm{R_1 = R_3}$ & $\bm{R_2}$ & $\bm{R_4}$  \\ \hline
		caso 1 & 2,5k & 25k & 10k \\
		caso 2 & 2,5k & 2,5k & 10k \\
		caso 3 & 25k & 2,5k & 100k \\
		\hline
	\end{tabular}
	\caption{Valores de resistencias para cada caso a analizar de los circuitos.}
	\label{casos}
\end{table}


\subsection{Respuesta en frecuencia del circuito}

\subsubsection{An\'alisis te\'orico} %%%%%%%

\subsubsection*{Configuraci\'on inversora}
Se calcula de forma te\'orica la ganancia del circuito \ref{c1}, 
considerando al amplificador operacional como ideal, es decir, 
con impedancia de entrada infinita e impedancia de salida nula. Se parte de las siguientes ecuaciones:
\begin{equation}
	\begin{cases}
		V_{out} = A_{vol}\cdot(V^+ - V^-) = -A_{vol} V^- \cdot \\
		V_{in} - R_1 \cdot I_1 = V^- \\
		V_{out} - R_2 \cdot I_2 = V^-\\ 
		I_3 = I_1 + I_2
	\end{cases}
	\label{ecsbase}
\end{equation}

Operando matem\'aticamente se obtiene la siguiente expresi\'on:

\begin{equation}
	\frac{V_{out}}{V_{in}} = - \frac{R_2/R_1}{1 + \frac{R_2/R_1}{A_{vol}} + \frac{R_2/R_3}{A_{vol}}}
	\label{c1vovi}
\end{equation}

\subsubsection*{Considerando $A_{vol} $finito}
 
En la tabla \ref{avolf} se puede ver el valor correspondiente a cada caso, obtenido al considerar 
$A_{vol} = A_{V0} = 100000$, cuyo valor fue sacado de la hoja de datos del LM324.

\begin{table}[h!]
	\centering
	\begin{tabular}{c c c}%
		\bfseries  & $V_{out}/V_{in}$ & $|V_{out}/V_{in}| (dB)$ \\ \hline
		caso 1 & $-9,998$ & $19,998$\\
		caso 2 & $ -1,00$ & $0,000$ \\
		caso 3 & $ -0,100$ & $-20,000$\\
		\hline
	\end{tabular}
	\caption{$V_{out}/V_{in}$ del circuito inversor considerando $A_{vol}$ finito.}
	\label{avolf}
\end{table}

En la tabla \ref{avolf} se observa que para la configuraci\'on del circuito \ref{c1} hay una ganancia de casi $20dB$ para el caso 1 de la distribuci\'on de resistencias, mientras que para el caso 2 no hay ganancia ni atenuaci\'on y para el caso 3 hay atenuaci\'on de $20dB$.  El resultado obtenido
no var\'ia con la frecuencia ya que el circuito es resistivo y se consider\'o,
para lo calculado reci\'en, que $A_{vol}$ no var\'ia con la frecuencia. Estos resultados
fueron obtenidos para cont\'inua ya que es el caso en el que $A_{vol}=A_{V0}$,
dado que: $A_{vol} = \frac{A_{V0}}{1+s/\omega_p} $ y cuando la frecuencia es cero,
este valor tiende a $A_{V0}$. Otra cosa que se observa en la tabla, son los signos de las relaciones de tensiones de salida respecto a las de entrada. El hecho de que los valores hayan dado negativos, corresponde con que se trata de un circuito de configuraci\'on inversora.


Por otro lado, reemplazando en la ecuaci\'on \ref{c1vovi} con los valores
 correspondientes de resistencias para cada uno de los tres casos
  (indicados en la tabla \ref{casos}) y considerando
   $A_{vol}(\omega)$ (con $s = j\omega$), se obtienen las siguientes expresiones:

Caso1:
\begin{equation}
	\frac{V_{out}}{V_{in}} = - \frac{7\cdot10^{12}}{2088,9 \cdot 10^3 s + 7 \cdot 10^{11}}
	\label{c1c1vovi}
\end{equation}

caso 2:
\begin{equation}
	\frac{V_{out}}{V_{in}} = - \frac{7 \cdot 10^{11}}{298,4 \cdot 10^{3} s +7 \cdot 10^{11}}
	\label{c1c2vovi}
\end{equation}

caso 3:
\begin{equation}
	\frac{V_{out}}{V_{in}} = - \frac{7 \cdot 10^{12}}{119,366 \cdot 10^{5} s +7 \cdot 10^{13}}
	\label{c1c3vovi}
\end{equation}

A continuaci\'on en los gr\'aficos \ref{c1voviTeoMod} y \ref{c1voviTeoPh} se observa c\'omo var\'ia la ganancia del circuito para los tres casos en funci\'on de la frecuencia.

\begin{figure}[H] %!ht
	\centering
	%\includegraphics[scale=0.45]{../EJ1/00GRAFICOS/teoricos/circ1voviw.png}
	\includegraphics[width=10cm,height=10cm,keepaspectratio]{../EJ1/00GRAFICOS/teoricos/circ1voviw.png}
	\caption{Configuración inversora - Comparaci\'on te\'orica del m\'odulo de $V_{out}/V_{in}$ para los tres casos.}
	\label{c1voviTeoMod}
\end{figure}

\begin{figure}[H] %!ht
	\centering
	\includegraphics[width=10cm,height=10cm,keepaspectratio]{../EJ1/00GRAFICOS/teoricos/circ1vovifasew.png}
	\caption{Configuración inversora - Comparaci\'on te\'orica de la fase de $V_{out}/V_{in}$ para los tres casos.}
	\label{c1voviTeoPh}
\end{figure}

El gr\'afico \ref{c1voviTeoMod} permite ver una caracter\'istica importante que diferencia a los tres casos de 
resistencias para este circuito: la ganancia a bajas frecuencias. Las tres configuraciones corresponden 
a filtros pasabajos. Si bien aten\'uan a altas frecuencias, 
tienen comportamientos diferentes en las frecuencias bajas. 
Aqu\'el con resistencias para el caso 1 presenta una ganancia de 20dB, 
mientras que el del caso 3 aten\'ua 20 dB. El circuito del caso 2, por el contrario, no gana ni aten\'ua 
en frecuencias bajas. A bajas frecuencias, lo que se observa en el gr\'afico \ref{c1voviTeoMod} coincide con los valores presentados previamente en la tabla \ref{avolf} (los cuales fueron hechos para el caso de frecuencia cero.


%%%%%%%%%%%%%%%%%%%%%%%%%%%%%%%%%%%%%%%%%%%%%%%%%%%%%%%%%%%%%%%%%%%%%%%%%
\subsubsection*{Configuraci\'on no inversora}

Para el circuito \ref{c2} se parte de las siguientes ecuaciones, similares a las del an\'alisis del circuito inversor. 

\begin{equation}
	\begin{cases}
		V_{out} = A_{vol}\cdot(V^+ - V^-) \\
		V_{in} - R_3 \cdot I_3 = V^+ \\
		V_{out} - R_2 \cdot I_2 = V^- \\
		I_4 = I_3\\
		I_2 = I_1
	\end{cases}
	\label{ecsbase2}
\end{equation}



Operando matem\'aticamente, se obtiene la siguiente expresi\'on:

\begin{equation}
	\frac{V_{out}}{V_{in}} =  \frac{A_{vol} \cdot R_4 \cdot (R_1 + R_2)}{(R_3 + R_4) \cdot (R_1 + R_2 + A_{vol} \cdot R_1)}
	\label{c2vovi}
\end{equation}

Al igual que para el circuito inversor, a continuaci\'on se muestran
los resultados obtenidos al evaluar la ecuaci\'on \ref{c2vovi} con las resistencias
de cada caso y considerando $A_{vol}$ finito.

\begin{table}[h!]
	\centering
	\begin{tabular}{c c c}%
		\bfseries  & $V_{out}/V_{in}$ &  $|V_{out}/V_{in}| (dB)$\\ \hline
		caso 1 & $8,799$ & $18,889$ \\
		caso 2 & $1,600$ & $4,082$\\
		caso 3 & $8,800$ & $18,889$\\
		\hline
	\end{tabular}
	\caption{$V_{out}/V_{in}$ del circuito inversor considerando $A_{vol}$ finito.}
	\label{avolf2}
\end{table}

A diferencia que para el circuito inversor, se puede notar que
 los valores de $V_{out}/V_{in}$ son positivos, lo cual concuerda con que se trata de un circuito no inversor. En este circuito los casos 1 y 3 presentan la misma ganancia de casi $19dB$ al considerar $A_{vol}$ finito, mientras que el caso 2 tambi\'en presenta ganancia, pero menor (de 4dB).


Por otro lado, reemplazando con los valores de resistencias correspondientes
 a cada caso y con $A_{vol}(\omega)$, se obtiene:

Caso1:-
\begin{equation}
	\frac{V_{out}}{V_{in}} = \frac{22 \cdot 10^8}{437,68 s + 25 \cdot 10^7}
	\label{c2c1vovi}
\end{equation}

caso 2:
\begin{equation}
	\frac{V_{out}}{V_{in}} = \frac{4 \cdot 10^8}{79,577 s + 25 \cdot 10^7}
	\label{c2c2vovi}
\end{equation}

caso 3:
\begin{equation}
	\frac{V_{out}}{V_{in}} = \frac{22 \cdot 10^8}{437,68 s + 25 \cdot 10^8}
	\label{c2c3vovi}
\end{equation}

En los gr\'aficos que se muestran a continuaci\'on (\ref{c2voviTeoMod} y \ref{c2voviTeoPh}) pueden verse los resultados de estos c\'alculos.

\begin{figure}[H] %!ht
\centering
\includegraphics[width=10cm,height=10cm,keepaspectratio]{../EJ1/00GRAFICOS/teoricos/circ2voviw.png}
\caption{Configuración no inversora - Comparaci\'on te\'orica del m\'odulo de$V_{out}/V_{in}$ de los tres casos.}
\label{c2voviTeoMod}
\end{figure}

\begin{figure}[H] %!ht
\centering
\includegraphics[width=10cm,height=10cm,keepaspectratio]{../EJ1/00GRAFICOS/teoricos/circ2vovifasew.png}
\caption{Configuración no inversora - Comparaci\'on te\'orica de la fase de $V_{out}/V_{in}$ de los tres casos.}
\label{c2voviTeoPh}
\end{figure}

Puede verse en el gr\'afico \ref{c2voviTeoMod} que a bajas frecuencias los valores de ganancias coinciden con los calculados para $A_{vol}$ finito. Al ser este gr\'afico en funci\'on de la frecuencia, permite observar que el no inversor tambi\'en se comporta como un filtro pasa bajos atenuando las altas frecuencias.






\subsubsection{Mediciones y resultados obtenidos} %%%%%%%

La respuesta en frecuencia es una herramienta de análisis útil para todo circuito. Se estimuló a los seis circuitos con una se\~nal senoidal de frecuencia variable, realizando un barrido por el espectro de frecuencias que consideramos relevante, en función a las simulaciones realizadas en LTSpice. Para cada valor de frecuencia se observó la diferencia de fase y la relación de amplitudes entre la se\~nal de entrada y la salida de los circuitos, conformando un diagrama de BODE completo. Se superpusieron dichas mediciones con las simulaciones, obteniendo los siguientes gráficos.
Cabe destacar que la medición estuvo altamente influida por fenómenos tales como el “crossover distortion” y el “slew rate” los cuales se desarrollarán a continuación.
\subsubsection*{Configuraci\'on inversora}
Se simul\'o y se midi\'o la ganancia para los tres casos del circuito \ref{c1} 
y a continuacio\'on se puede ver la diferencia entre sus resultados y los 
de las ecuaciones \ref{c1c1vovi}, \ref{c1c2vovi} y \ref{c1c3vovi}; correspondientes a 
la ganancia calculada de forma te\'orica y considerando al amplificador operacional como ideal.


\begin{figure}[H] %!ht
	\centering
	\includegraphics[width=10cm,height=10cm,keepaspectratio]{../EJ1/00GRAFICOS/c1c1/c1c1voviMod.png}
	\caption{Configuración inversora -  Caso 1 - Módulo de $V_{out}/V_{in}$}
	\label{c1c1voviM}
\end{figure}

\begin{figure}[H] %!ht
	\centering
	\includegraphics[width=10cm,height=10cm,keepaspectratio]{../EJ1/00GRAFICOS/c1c1/c1c1voviFASE.png}
	\caption{Configuración inversora - Caso 1 - Fase de $V_{out}/V_{in}$}
	\label{c1c1voviP}
\end{figure}

\begin{figure}[H] %!ht
	\centering
	\includegraphics[width=10cm,height=10cm,keepaspectratio]{../EJ1/00GRAFICOS/c1c2/c1c2voviMod.png}
	\caption{Configuración inversora - Caso 2 - Módulo de $V_{out}/V_{in}$}
	\label{c1c2voviM}
\end{figure}

\begin{figure}[H] %!ht
	\centering
	\includegraphics[width=10cm,height=10cm,keepaspectratio]{../EJ1/00GRAFICOS/c1c2/c1c2voviFASE.png}
	\caption{Configuración inversora - Caso 2 - Fase de $V_{out}/V_{in}$ }
	\label{c1c2voviP}
\end{figure}

\begin{figure}[H] %!ht
	\centering
	\includegraphics[width=10cm,height=10cm,keepaspectratio]{../EJ1/00GRAFICOS/c1c3/c1c3voviMod.png}
	\caption{Configuración inversora - Caso 3 - M\'odulo de$V_{out}/V_{in}$}	
	\label{c1c3voviM}
\end{figure}

\begin{figure}[H] %!ht
	\centering
	\includegraphics[width=10cm,height=10cm,keepaspectratio]{../EJ1/00GRAFICOS/c1c3/c1c3voviFASE.png}
	\caption{Configuración inversora - Fase de $V_{out}/V_{in}$}
	\label{c1c3voviP}
\end{figure}

\subsubsection*{Configuraci\'on no inversora}

\begin{figure}[H] %!ht
	\centering
	\includegraphics[width=10cm,height=10cm,keepaspectratio]{../EJ1/00GRAFICOS/c2c1/c2c1voviMod.png}
	\caption{Configuración no inversora - Caso 1 -  M\'odulo de $V_{out}/V_{in}$}
	\label{c2c1voviM}
\end{figure}

\begin{figure}[H] %!ht
	\centering
	\includegraphics[width=10cm,height=10cm,keepaspectratio]{../EJ1/00GRAFICOS/c2c1/c2c1voviFASE.png}
	\caption{Configuración no inversora - Caso 1 - Fase de $V_{out}/V_{in}$}
	\label{c2c1voviP}
\end{figure}

\begin{figure}[H] %!ht
	\centering
	\includegraphics[width=10cm,height=10cm,keepaspectratio]{../EJ1/00GRAFICOS/c2c2/c2c2voviMod.png}
	\caption{Configuración no inversora - Caso 2 - M\'odulo de $V_{out}/V_{in}$}
	\label{c2c2voviM}
\end{figure}

\begin{figure}[H] %!ht
	\centering
	\includegraphics[width=10cm,height=10cm,keepaspectratio]{../EJ1/00GRAFICOS/c2c2/c2c2voviFASE.png}
	\caption{Configuración no inversora - Caso 2 - Fase de $V_{out}/V_{in}$}
	\label{c2c2voviP}
\end{figure}

%\begin{figure}[H] %!ht
%	\centering
%	\includegraphics[width=10cm,height=10cm,keepaspectratio]{../EJ1/00GRAFICOS/c2c3/c2c3voviMod.png}
%	\caption{Configuración no inversora - Caso 3 - M\'odulo de $V_{out}/V_{in}$}	
%	\label{c2c3voviM}
%\end{figure}

\begin{figure}[H] %!ht
	\centering
	\includegraphics[width=10cm,height=10cm,keepaspectratio]{../EJ1/00GRAFICOS/c2c3/c2c3voviFASE.png}
	\caption{Configuración no inversora - Caso 3 - Fase de $V_{out}/V_{in}$}
	\label{c2c3voviP}
\end{figure}




\subsection{Impedancia de entrada del circuito} %%%%%


\subsubsection{An\'alisis te\'orico} %%%%%%

\subsubsection*{Configuraci\'on inversora}
Se hicieron distintos c\'alculos que permitieron obtener expresiones distintas para la impedancia de entrada del circuito.

Es importante mencionar que en primer lugar se consider\'o al amplificador operacional como ideal. Si bien las caracter\'isticas de dicha situaci\'on fueron mencionadas previamente, se considera relevante hacer unas breves aclaraciones para comprender el resultado de este an\'alisis. La impedancia entre los bornes de entrada del amplificador operacional fue, por lo tanto, tomada como infinita (circuito abierto); mientras que la impedancia interna a la salida del amplificador operacional fue considerada como cero (cable). Dado que en el caso ideal del amplificador operacional hay una masa virtual en $V^-$ y que la entrada $V^+$ est\'a f\'isicamente conectada a Tierra, la fuente interna que se encuentra en serie con la impedancia de salida vale cero ya que depende de la diferencia de tensi\'on entre $V^+$ y $V^-$. Entonces, partiendo de las ecuaciones \ref{ecsbase} y operando matem\'aticamente se obtiene la siguiente expresi\'on:

\begin{equation}
	Z_{in} =  \frac{A_{vol} \cdot R_1 + R_1 + R_2}{1 + A_{vol}}
	\label{zint}
\end{equation}

Reemplazando con los valores de resistencias correspondientes a cada caso (Ver tabla \ref{casos}), se obtiene una impedancia de entrada distinta para cada uno:

Caso1:
\begin{equation}
	Z_{in} =  \frac{437,68 s + 28 \cdot 10^7}{1,59 \cdot 10^{-2} s + 11,2 * 10^4}
	\label{c1c1zint}
\end{equation}

caso 2:
\begin{equation}
	Z_{in} =  \frac{79,58 s + 28 \cdot 10^7}{1,59 \cdot 10^{-2} s + 11,2 \cdot 10^4}
	\label{c1c2zint}
\end{equation}

caso 3:
\begin{equation}
	Z_{in} =  \frac{437,68 s + 28 \cdot 10^8}{1,59 \cdot 10^{-2} s + 11,2 \cdot 10^4}
	\label{c1c3zint}
\end{equation}


Dado que luego se llevar\'ian a cabo mediciones para contrastar los resultados con el c\'alculo te\'orico, se decidi\'o buscar la expresi\'on correspondiente a la $Z_{in}$ que incluyera una punta del osciloscopio, es decir, se calcul\'o la impedancia que ser\'ia vista idealmente al utilizar el osciloscopio. Para esto, se le agreg\'o en paralelo el modelo equivalente a una punta X10 (la empleada) al resultado obtenido previamente de la $Z_{in}$. Dicho modelo consiste en una resistencia de $10M\Omega$ en paralelo con un capacitor de $12pF$. As\'i se obtuvo la siguiente expresi\'on:

\begin{equation}
	Z_{in}\rvert_{c/punta} = 8.33 \cdot 10^17\cdot (A \cdot R1 + R1 + R2)/(8.33 \cdot 10^17 \cdot A + (10^7 s + 8,33 \cdot 10^10) \cdot (A \cdot R1 + R1 + R2) + 8.33 \cdot 10^17)
	\label{zinp}
\end{equation}

Evaluando para cada uno de los casos indicados en la tabla \ref{casos}, se llega a las siguientes expresiones para la impedancia de entrada incluyendo la punta X10 del osciloscopio $Z_{in}\rvert_{c/punta}$:

Caso 1:
\begin{equation}
	Z_{in}\rvert_{c/punta} = \frac{3,65 \cdot 10^{20} s + 2,33 \cdot 10^{26}}{437,68 \cdot 10^7 s^2 + 1,61 \cdot 10^{26} s + 9,34 \cdot 10^{22}}
	\label{c1c1zinp}
\end{equation}

Caso 2:
\begin{equation}
	Z_{in}\rvert_{c/punta} = \frac{6,63 \cdot 10^{19} s + 2,33 \cdot 10^{26}}{79,58 \cdot 10^7 s^2 + 1,61 \cdot 10^{16} s + 9,34 \cdot 10^{22}}
	\label{c1c2zinp}
\end{equation}

Caso 3:
\begin{equation}
	Z_{in}\rvert_{c/punta} = \frac{3,65 \cdot 10^{20} s + 2,33 \cdot 10^{27}}{437,68 \cdot 10^7 s^2 + 4,13 \cdot 10^{16} s + 9,36 \cdot 10^{22}}
	\label{c1c3zinp}
\end{equation}


A continuaci\'on, en los gr\'aficos \ref{c1zintm} y \ref{c1zintp} se muestra la impedancia de entrada del circuito calculada de forma te\'orica con y sin punta del osciloscopio para los tres casos de la tabla \ref{casos}:

\begin{figure}[H] %!ht
	\centering
	\includegraphics[width=10cm,height=10cm,keepaspectratio]{../EJ1/00GRAFICOS/teoricos/c1zinm.png}
	\caption{Configuración inversora - M\'odulo de $Z_{in}$ calculada de forma te\'orica con y sin punta del osciloscopio.}
	\label{c1zintm}
\end{figure}

\begin{figure}[H] %!ht
	\centering
	\includegraphics[width=10cm,height=10cm,keepaspectratio]{../EJ1/00GRAFICOS/teoricos/circ1zinfases.png}
	\caption{Configuración inversora - Fase de $Z_{in}$ calculada de forma te\'orica con y sin la punta del osciloscopio.}
	\label{c1zintp}
\end{figure}
%hablar de las diferencias entre cada caso

\subsubsection*{Configuraci\'on no inversora}

Los c\'alculos te\'oricos que se llevaron a cabo para la impedancia de entrada del circuitode configuraci\'on no inversora \ref{c2} son an\'alogos a los realizados para el circuito inversor \ref{c1}. La expresi\'on \ref{c2zint} corresponde a c\'alculo te\'orico que se hizo sin tener en cuenta la punta del osciloscopio:

\begin{equation}
	Z_{in} =  R_3 + R_4
	\label{c2zint}
\end{equation}

Reemplazando con los valores de resistencias correspondientes a cada caso (Ver tabla \ref{casos}), se obtienen las siguientes impedancia de entrada:

Casos 1 y 2:
\begin{equation}
	Z_{in} =  12,5k\Omega
	\label{c2c1zint}
\end{equation}

caso 3:
\begin{equation}
	Z_{in} =  125k\Omega
	\label{c3c3zint}
\end{equation}

Como el circuito es resistivo y la expresi\'on \ref{c2zint} no depende de $A_{vol}$, la impedancia de entrada se mantiene constante para todas las frecuencias. Sin embargo, a continuaci\'on se muestra la expresi\'on obtenida al hacer el c\'alculo considerando la punta del osciloscopio a la entrada, lo cual permite ver que de esa forma deja de mantenerse constante la impedancia previamente calculada.


\begin{equation}
	Z_{in}\rvert_{c/punta} = 8,33 \cdot 10^{17} \cdot (R3 + R4)/((R3 + R4) \cdot (10^7 s + 8,33 \cdot 10^{10}) + 8,33 \cdot 10^{17})
	\label{zinp}
\end{equation}

Si bien la expresi\'on del c\'alculo con la punta del osciloscopio, \ref{zinp}, sigue sin depender de $A_{vol}$, var\'ia con la frecuencia ya que el modelo de la punta tiene un capacitor.

Evaluando para cada uno de los casos indicados en la tabla \ref{casos}, se llega a las siguientes expresiones para la impedancia de entrada incluyendo la punta X10 del osciloscopio $Z_{in}\rvert_{c/punta}$:

Casos 1 y 2:
\begin{equation}
	Z_{in}\rvert_{c/punta} = \frac{1,04 \cdot 10^22}{1,25 \cdot 10^11 + 8.34 \cdot 10^17}
	\label{c2c1zinp}
\end{equation}

Caso 3:
\begin{equation}
	Z_{in}\rvert_{c/punta} = \frac{1,04 \cdot 10^23}{1,25 \cdot 10^12 + 8,43 \cdot 10^17}
	\label{c2c3zinp}
\end{equation}


A continuaci\'on, en los gr\'aficos \ref{c2zintm} y \ref{c2zintp} se muestra la impedancia de entrada del circuito calculada de forma te\'orica con y sin punta del osciloscopio para los tres casos de la tabla \ref{casos}:

\begin{figure}[H] %!ht
	\centering
	\includegraphics[width=10cm,height=10cm,keepaspectratio]{../EJ1/00GRAFICOS/teoricos/circ2zinm.png}
	\caption{Configuración inversora - M\'odulo de $Z_{in}$ calculada de forma te\'orica con y sin punta del osciloscopio.}
	\label{c2zintm}
\end{figure}

\begin{figure}[H] %!ht
	\centering
	\includegraphics[width=10cm,height=10cm,keepaspectratio]{../EJ1/00GRAFICOS/teoricos/circ2zinfase.png}
	\caption{Configuración inversora - Fase de $Z_{in}$ calculada de forma te\'orica con y sin la punta del osciloscopio.}
	\label{c2zintp}
\end{figure}
%hablar de las diferencias entre cada caso

\subsubsection{Mediciones y resultados obtenidos} %%%%%%

\subsubsection*{Configuraci\'on inversora}
Para medir la impedancia de entrada del circuito en funci\'on de la frecuencia, deb\'iamos hacer el cociente $V_{in}/I_{in}$. Si bien se puede medir la tensi\'on de entrada al circuito de forma directa con el osciloscopio, no es tan sensillo obtener la corriente que entra al circuito, ya que el osciloscopio mide tensiones y no corrientes. Se busc\'o una resistencia $R_L$ cuyo valor comerical fuera lo m\'as parecido posible (igual o el primero mayor) al valor obtenido en el c\'alculo te\'rico para cada uno de los casos de resistencias. Se coloc\'o dicha resistencia en serie al generador, a la entrada del circuito. Luego se midi\'o la ca\'ida de tensi\'on sobre ella, ya que al dividirla por el valor de la $R_L$ colocada se obtendr\'ia la corriente de entrada al circuito $I_{in}$. El criterio de buscar una resistencia similar al valor calculado de $Z_{in}$ surge de que si se pusiese una resistencia muy chica, la diferencia entre las tensiones medidas sobre sus bornes ser\'ia muy chica (aumentando incertidumbre) y si se colocase una resistencia muy grande, la tensi\'on que caer\'ia ser\'ia mucho mayor a la que caer\'ia en el circuito, haciendo que la tensi\'on luego de la resistencia sea muy chica (se podr\'ia acercar al nivel de ruido) y que la diferencia de tensi\'on entre sus bornes tienda a la tensi\'on entregada por el generador. Por eso se consider\'o \'optimo que la resistencia tenga un valor similar al calculado de forma te\'orica y en caso de no conseguir el mismo valor, prefiri\'endose un valor mayor y no menor.

\begin{figure}[H] %!ht
	\centering
	\includegraphics[width=10cm,height=10cm,keepaspectratio]{../EJ1/00GRAFICOS/c1c1/c1c1ZINpunta.png}
	\caption{Configuración inversora - Caso 1 - M\'odulo de $Z_{in}$}
	\label{c1c1zinM}
\end{figure}

\begin{figure}[H] %!ht
	\centering
	\includegraphics[width=10cm,height=10cm,keepaspectratio]{../EJ1/00GRAFICOS/c1c1/c1c1zinFASE.png}
	\caption{Configuración inversora - Caso 1 - Fase de $Z_{in}$ }
	\label{c1c1zinP}
\end{figure}

\begin{figure}[H] %!ht
	\centering
	\includegraphics[width=10cm,height=10cm,keepaspectratio]{../EJ1/00GRAFICOS/c1c2/c1c2ZINpunta.png}
	\caption{Configuración inversora - Caso 2 - M\'odulo de $Z_{in}$}
	\label{c1c2zinM}
\end{figure}

\begin{figure}[H] %!ht
	\centering
	\includegraphics[width=10cm,height=10cm,keepaspectratio]{../EJ1/00GRAFICOS/c1c2/c1c2zinFASE.png}
	\caption{Configuración inversora - Caso 2 - Fase de $Z_{in}$}
	\label{c1c2zinP}
\end{figure}

\begin{figure}[H] %!ht
	\centering
	\includegraphics[width=10cm,height=10cm,keepaspectratio]{../EJ1/00GRAFICOS/c1c3/c1c3ZINpunta.png}
	\caption{Configuración inversora - Caso 3 - M\'odulo de $Z_{in}$}
	\label{c1c3zinM}
\end{figure}

\begin{figure}[H] %!ht
	\centering
	\includegraphics[width=10cm,height=10cm,keepaspectratio]{../EJ1/00GRAFICOS/c1c3/c1c3zinFASE.png}
	\caption{Configuración inversora - Caso 3 - Fase de $Z_{in}$}
	\label{c1c3zinP}
\end{figure}

\subsubsection*{Configuraci\'on no inversora}
\begin{figure}[H] %!ht
	\centering
	\includegraphics[width=10cm,height=10cm,keepaspectratio]{../EJ1/00GRAFICOS/c2c1/c2c1ZINpunta.png}
	\caption{Configuración no inversora - Caso 1 - M\'odulo de $Z_{in}$}
	\label{c2c1zinM}
\end{figure}

\begin{figure}[H] %!ht
	\centering
	\includegraphics[width=10cm,height=10cm,keepaspectratio]{../EJ1/00GRAFICOS/c2c1/c2c1zinFASE.png}
	\caption{Configuración no inversora - Caso 1 - Fase de $Z_{in}$}
	\label{c2c1zinP}
\end{figure}

\begin{figure}[H] %!ht
	\centering
	\includegraphics[width=10cm,height=10cm,keepaspectratio]{../EJ1/00GRAFICOS/c2c2/c2c2ZINpunta.png}
	\caption{Configuración no inversora - Caso 2 - M\'odulo de $Z_{in}$}
	\label{c2c2zinM}
\end{figure}

\begin{figure}[H] %!ht
	\centering
	\includegraphics[width=10cm,height=10cm,keepaspectratio]{../EJ1/00GRAFICOS/c2c2/c2c2zinFASE.png}
	\caption{Configuración no inversora - Caso 2 - Fase de $Z_{in}$}
	\label{c2c2zinP}
\end{figure}

%\begin{figure}[H] %!ht
%	\centering
%	\includegraphics[width=10cm,height=10cm,keepaspectratio]{../EJ1/00GRAFICOS/c2c3/c2c3ZINpunta.png}
%	\caption{Configuración no inversora - Caso 3 - M\'odulo de $Z_{in}$}
%	\label{c2c3zinM}
%\end{figure}

\begin{figure}[H] %!ht
	\centering
	\includegraphics[width=10cm,height=10cm,keepaspectratio]{../EJ1/00GRAFICOS/c2c3/c2c3zinFASE.png}
	\caption{Configuración no inversora - Caso 3 - Fase de $V_{out}/V_{in}$}
	\label{c2c3zinP}
\end{figure}





\subsection{DC Sweep desde $-V_{CC}$ hasta $V_{CC}$}
%Buscar valor maximo en hoja de datos vcc aprox 17 creo.BUSCAR voutMAX!!!
Dado que se nos pidi\'o alimentar al amplificador operacional con $V_{CC} = \pm 15V$, un DC Sweep desde $-V_{CC}$ hasta $V_{CC}$ requerir\'ia $30V_{pp}$ del generador de se\~nales. Una limitaci\'on de los generadores del laboratorio es que alcanzan un m\'aximo de $20V_{pp}$, por lo que no podr\'iamos llevar a cabo las mediciones generando una rampa en el rango de tesniones mencionado. La desici\'on tomada para lograr lo pedido fue, en el dise\~no del circuito, agregarle una etapa previa de amplificaci\'on utilizando otro amplificador operacional. 
El amplificador operacional no permite amplificar mas de un valor determinado, y por lo tanto no hay forma de llegar exactamente a -15V y a 15V a la entrada del circuito ya que su tensi\'on de entrada es la salida del amplificador operacional empleado en la etapa previa de amplificaci\'on de la se\~nal del generador.

\subsubsection*{Configuraci\'on inversora} %%%%%%%%%%%%%%%%%%%%%%%%%%

\begin{figure}[H] %!ht
	\centering
	\includegraphics[width=10cm,height=10cm,keepaspectratio]{../EJ1/00GRAFICOS/c1dcs/c1c1dcs.png}
	\caption{DC Sweep del circuito inversor, caso 1.}
	\label{c1c1dcs}
\end{figure}

\begin{figure}[H] %!ht
	\centering
	\includegraphics[width=10cm,height=10cm,keepaspectratio]{../EJ1/00GRAFICOS/c1dcs/c1c2dcs.png}
	\caption{DC Sweep del circuito inversor, caso 2.}
	\label{c1c2dcs}
\end{figure}

\begin{figure}[H] %!ht
	\centering
	\includegraphics[width=10cm,height=10cm,keepaspectratio]{../EJ1/00GRAFICOS/c1dcs/c1c3dcs.png}
	\caption{DC Sweep del circuito inversor, caso 3.}
	\label{c1c2dcs}
\end{figure}

\subsubsection*{Configuraci\'on no inversora} %%%%%%%%%%%%%%%%%%%%%%%%%%

\begin{figure}[H] %!ht
	\centering
	\includegraphics[width=10cm,height=10cm,keepaspectratio]{../EJ1/00GRAFICOS/c2dcs/c2c1dcs.png}
	\caption{DC Sweep del circuito no inversor, caso 1.}
	\label{c2c1dcs}
\end{figure}

\begin{figure}[H] %!ht
	\centering
	\includegraphics[width=10cm,height=10cm,keepaspectratio]{../EJ1/00GRAFICOS/c2dcs/c2c2dcs.png}
	\caption{DC Sweep del circuito no inversor, caso 2.}
	\label{c2c2dcs}
\end{figure}

\begin{figure}[H] %!ht
	\centering
	\includegraphics[width=10cm,height=10cm,keepaspectratio]{../EJ1/00GRAFICOS/c2dcs/c2c3dcs.png}
	\caption{DC Sweep del circuito no inversor, caso 3.}
	\label{c2c2dcs}
\end{figure}
%BUSCAR VALORES!!!!!!
\subsection{Fen\'omenos que afectan al comportamiento de los circuitos (limitaciones)} %VER SI DEJAR PARA C/CASO O AL FINAL, for BOTH, o al ppio introductoriamente
	\subsubsection{Efecto de slew rate (SR)}
	El Slew Rate es un fen\'omeno que se produce a la salida de un amplificador operacional, 
	por el cual la se\~nal de salida se ve alterada respecto de la se\~nal de entrada. 
	Esta alteraci\'on se produce en aquellos segmentos temporales donde la derivada de 
	la se\~nal de salida $te\'orica$ (es decir, la que seg\'un el modelo se deber\'ia observar 
	a la salida) supera un determinado valor, especificado por el fabricante del circuito 
	integrado en su respectiva hoja de datos. En esos casos, la pendiente de la se\~nal se ver\'a 
	limitada a la indicada por este coeficiente. A continuaci\'on se muestra la ecuai\'on que define
	al slew rate (SR):

	\begin{equation}
		SR = m\'ax\bigg\{\frac{ dV_{out}}{dt}\bigg\}
		\label{srr}
	\end{equation}

	Para el caso del circuito integrado utilizado (LM324 de Texas Instruments) 
	el valor del slew rate es de $0.5 \frac{V}{\mu s}$. Esto fue obtenido a partir
	del gr\'afico \ref{srdatasheet} presentado por el fabricante. 

	\begin{figure}[H] %!ht
		\centering
		\includegraphics[width=10cm,height=10cm,keepaspectratio]{../EJ1/00GRAFICOS/SRdatasheet.png}
		\caption{Gr\'afico de la hoja de datos de la que se obtuvo el valor del slew rate.}
		\label{srdatasheet}
	\end{figure}

	El valor sale de calcular la pendiente de la tensi\'on de salida como respuesta a la se\~nal cuadrada de entrada, a partir del cociente entre $\Delta V$ y $\delta t$ como se ve en la figura \ref{srt}.

	\begin{figure}[H] %!ht
		\centering
		\includegraphics[width=10cm,height=10cm,keepaspectratio]{../EJ1/00GRAFICOS/srt.png}
		\caption{Detalles del gr\'afico para el c\'alculo del slew rate.}
		\label{srt}
	\end{figure}

	A continuaci\'on se presenta una imagen del osciloscopio mostrando el efecto que fue observando en la pr\'actica.

	\begin{figure}[H] %!ht
		\centering
		\includegraphics[width=10cm,height=10cm,keepaspectratio]{../EJ1/00GRAFICOS/srm.png}
		\caption{Efecto de slew rate observado en el osciloscopio.}
		\label{srm}
	\end{figure}

	Se midi\'o el slew rate alimentando uno de los circuitos con una se\~nal senoidal de 3Vpp y 800KHz;
	ya que en esas condiciones a la salida se pod\'ia observar dicho efecto. Se colocaron los cursores en 
	una posici\'on que permitiera calcular la pendiente y se obtuvo:
	\begin{equation}
		\begin{cases}
			\Delta X = 466 ns \\
			\Delta Y = 246 mV
		\end{cases}
		\label{srmedicion}
	\end{equation}

	y por lo tanto, $SR = 0,526 \frac{V}{\mu s}$, lo cual coincide mucho con el valor de la hoja de datos.
	

	Esta distorsión en la se\~nal obliga a tener ciertas consideraciones en las mediciones. 
	Por ejemplo, en el caso de la medici\'on de respuesta en frecuencia de los circuitos, 
	en donde es de vital importancia conocer el valor pico a pico de la se\~nal de salida, 
	se debe cuidar que el producto de la frecuencia de excitaci\'on y la amplitud de salida no
	 exceda al coeficiente de slew rate, como mínimo. Caso contrario la señal senoidal tenderá 
	 a ser triangular, con lo cual las mediciones no serán útiles. Como consecuencia de esto, 
	 a medida que se aumenta la frecuencia de la entrada se debe moderar la amplitud de esta 
	 señal, para evitar la aparición de slew rate en la salida.

	 Según se pudo investigar, el origen de esta limitación se encuentra en el agregado de un 
	 capacitor de compensación en el circuito interno del operacional, el cual se utiliza 
	 para modificar la respuesta en frecuencia del mismo amplificador. Por otro lado, en 
	 la práctica se desprecie este efecto en la mayoría de los casos, dado el alto coeficiente 
	 de slew rate que poseen algunos circuitos integrados en la actualidad. 



	\subsubsection{Distorsi\'on de cruce por cero (Cross-Over Distortion)}
	Otro fenómeno para tener en cuenta al momento de efectuar mediciones es el
	 denominado “crossover distortion”. Este efecto se hace presente cuando la tensión 
	 de entradas se aproxima a cero, u oscila en un rango aproximado de -0.7V a 0.7V.
	  Estas últimas dos tensiones son justamente aquellas que accionan un diodo de silicio, 
	  y análogamente describen las tensiones base-emisor de un transistor BJT.
Podemos modelar a la salida de un amplificador operacional como dos transistores 
(uno NPN y otro PNP) dispuestos como se muestra en la siguiente figura \ref{cross}. Esto es tenido en cuenta 
ya que internamente el amplificador operacional tiene transistores que generan este comportamiento.

\begin{figure}[H] %!ht
	\centering
	\includegraphics[width=10cm,height=10cm,keepaspectratio]{../EJ1/00GRAFICOS/cross.png}
	\caption{Efecto de crossover distortion.}
	\label{cross}
\end{figure}

Cuando la tensi\'on de input se acerca a los rangos antes mencionados, 
un transistor se pone en modo de corte y el otro en modo saturaci\'on. 
En esta conmutaci\'on entre transistores se produce un efecto alineal, 
dada la alinealidad de la curva característica de salida de un transistor. 
De esta forma, la señal de salida no $sigue$ en forma a la se\~nal de entrada, 
por lo que se produce una ruptura en el lazo de alimentación en el circuito. 
Luego, obtenemos un recorte de la señal de salida, como se muestra en la figura anterior. 
Esto nuevamente tiene impacto sobre la medici\'on de la señal, dado que esta no refleja 
el comportamiento esperado del circuito. Para compensar este efecto se encontró que es 
especialmente \'util superponer un nivel de continua sobre la señal que se inyecte al circuito. 
Esto ayuda a disminuir este efecto, en la mayor\'ia de los casos. Dicho nivel de offset en la se\~nal fue agregado de forma totalmente empírica a cada medición, en función de lo reflejado en la salida en la pantalla del osciloscopio.
A continuación se muestran ejemplos de esta distorsi\'on en las sucesivas mediciones practicadas.


	\subsubsection{Gain Bandwidth Product (GBP)}
	El gain-bandwidth product (o GBP por sus siglas) es el producto entre el ancho de banda 
	y la ganancia máxima que presenta el amplificador operacional. El ancho de banda est\'a
	definido por la frecuencia para la cual hay una ca\'ida de 3dB de la tensi\'on de salida
	respecto a la de entrada del circuito. Esto puede ser visto facilmente en la respuesta
	en frecuencia de un circuito en la ubicaci\'on del primer polo. Los amplificadores
	operacionaes tienen un polo dominante. Dado que se comportan como un pasa bajos, dicho 
	polo es aqu\'el que determina la frecuencia para la cual hay atenuaci\'on de 3dB. El polo 
	dominante depende de los componentes internos del operacional, por lo que cada amplificador
	operacional tiene un polo dominante diferente, y por lo tanto un ancho de banda distinto. Esto
	significa que seg\'un la frecuencia a la cual se desea trabajar, se deber\'a elegir un 
	amplificador operacional que defina un ancho de banda que abarque la regi\'on de frecuencias 
	requerida en el dise\~no del circuito, ya que influyer\'a en la respuesta en frecuencia
	del circuito.
	

	\subsubsection{saturación}
	Según el modelo simplificado e ideal de un amplificador operacional, lo único que rige a la señal de salida es la ganancia del componente respecto de la señal de entrada, presentando un comportamiento netamente lineal. Es decir, si tenemos una configuración de amplificador cuya ganancia es 1000, e inyectamos una señal cuyo valor pico a pico es de 1V, la salida será de igual forma a la entrada con un valor pico a pico de 1000V.
En la práctica el circuito integrado que contiene a los amplificadores operacionales está alimentado por una fuente de corriente continua constante (en este caso simétrica) que es la que entrega la potencia en el momento de amplificar una señal. Esta aproximación a la realidad supone una restricción al circuito, siendo esta que los valores máximos y mínimos de la señal de salida nunca podrán superar a los valores máximos y mínimos de alimentación (+Vcc, -Vcc), respectivamente. Como consecuencia de esto, la salida del circuito entrará en estado de “saturación” cuando la tensión de salida sobrepase los límites de la alimentación, lo que deriva en un comportamiento alineal de la salida. 

Al llevar a cabo el DC sweep del cual se habl\'o previamente, se pudo observar que no hab\'ia forma de llegar realmente a los $\pm 15V$ a la salida,
que es la tensi\'on de $V_{cc}$ con la que se alimentaba al integrado, sino que la saturaci\'on se daba cerca de los 14V. Esto se debe a que en la realidad
la saturaci\'on ocurre un poco antes debido a caracter\'isticas de c\'omo est\'a hecho internamente el integrado.
Por lo tanto, esto se puede deber a alguna ca\'ida de tensi\'on interna dentro del amplificador operacional no predicha en el modelo teórico (como por ejemplo caídas de tensión para la alimentación de los transistores que componen la lógica interna del amplificador).
\subsection{Condiciones de comportamiento lineal del circuito}

\subsubsection{An\'alisis te\'orico}

La tensión de entrada máxima del circuito está limitada principalmente 
por el slew rate slew rate y la saturaci\'on. 

\subsubsection*{Influenccia del slew rate en $V_{in_{max}}$}

Partiendo de:
\begin{equation}
\begin{cases}
	SR = m\'ax\bigg\{\frac{ dV_{out}}{dt}\bigg\} \\
	V_{in} (f, t) = V_{in_{max}} \cdot sin(2\pi f t) \\
	V_{out} (f, t) = \rvert H(f)\rvert \cdot V_{in_{max}} \cdot sin(2 \pi f t)
\end{cases}
\label{srecs}
\end{equation}
 
Siendo $SR$ el slew rate, $V_{in}$ y $V_{out}$ las se\~nales de entrada y de salida respectivamente y $\rvert H(f)\rvert = V_{out}/V_{in}$ la ganancia del circuito.


\begin{equation}
	\frac{dV_{out}}{dt} = \rvert H(f)\rvert V_{in_{max}} 2 \pi f cos(2 \pi f t)
\label{deriv}
\end{equation}

Maximizando la ecuaci\'on \ref{deriv} se obtiene que:

\begin{equation}
	SR = m\'ax\bigg\{\frac{dV_{out}}{dt}\bigg\} = \rvert H(f)\rvert 2 \pi f V_{in_{max}} 
\label{max}
\end{equation}

Despejando de la ecuaci\'on \ref{max}:

\begin{equation}
	 V_{in_{max}}  = \frac{SR}{\rvert H(f)\rvert 2\pi f}
	 \label{vinmax}
\end{equation}

El valor de SR, para el c\'alculo te\'orico, fue sacado de hojas de datos del amplificador operacional LM324 de Texas Instrument\footnote{Hoja de datos del operacional LM324: http://www.ti.com/lit/ds/symlink/lm324-n.pdf}.
Se encontr\'o que $SR = 0,5 \frac{V}{\mu s}$. Reemplazando con este valor y con los valores correspondientes de resistencias:

%%%%%%%%%%%%%%%%%%%%%%%%%%%%%%%%%%%%%%%%%%%%%%%%%%%%%%%%%%%
\subsubsection*{- Configuraci\'on inversora}

\begin{table}[h!]
	\centering
	\begin{tabular}{c c c }%
		\bfseries  & $\bm{V_{in_{max}}}$ \\ \hline
		caso 1 & $\frac{3.38 \cdot 10^{-9} \sqrt{1.87 \cdot 10^{15} f^{2} + 8.68 \cdot 10^{24}}}{f}$\\ \\
		caso 2 & $\frac{1.86 \cdot 10^{-8} \sqrt{6.17 \cdot 10^{13} f^{2} + 8.68 \cdot 10^{24}}}{f}$ \\ \\
		caso 3 & $\frac{3.38 \cdot 10^{-9} \sqrt{1.87 \cdot 10^{15} f^{2} + 8.68 \cdot 10^{26}}}{f}$ \\
		\hline
	\end{tabular}
	\caption{Circuito inversor: Tensi\'on de entrada m\'axima, limitada por slew rate.}
	\label{vin_max}
\end{table}

\begin{figure}[H] %!ht
	\centering
	\includegraphics[width=10cm,height=10cm,keepaspectratio]{../EJ1/00GRAFICOS/teoricos/c1sr.png}
	\caption{Circuito inversor: Tensi\'on de entrada m\'axima limitada por slew rate.}
	\label{c1sr}
\end{figure}

\subsubsection*{- Configuraci\'on no inversora}

\begin{table}[h!]
	\centering
	\begin{tabular}{c c c }%
		\bfseries  & $\bm{V_{in_{max}}}$ \\ \hline
		caso 1 & $\frac{3.38 \cdot 10^{-9} \sqrt{1.87 \cdot 10^{15} f^{2} + 8.68 \cdot 10^{24}}}{f}$ \\ \\
		caso 2 & $\frac{1.86 \cdot 10^{-8} \sqrt{6.17 \cdot 10^{13}f^{2} + 8.68 \cdot 10^{24}}}{f}$ \\ \\
		caso 3 & $\frac{3.38 \cdot 10^{-9} \sqrt{1.87 \cdot 10^{15} f^{2} + 8.68 \cdot 10^{26}}}{f}$ \\
		\hline
	\end{tabular}
	\caption{Circuito no inversor: Tensi\'on de entrada m\'axima limitada por slew rate.}
	\label{vin_max}
\end{table}

\begin{figure}[H] %!ht
	\centering
	\includegraphics[width=10cm,height=10cm,keepaspectratio]{../EJ1/00GRAFICOS/teoricos/c2sr.png}
	\caption{Circuito no inversor: Tensi\'on de entrada m\'axima limitada por slew rate.}
	\label{c2sr}
\end{figure}

%%%%%%%%%%%%%%%%%%%%%%%%%%%%%%%%%%%%%%%%%%%%%%%%%%%%
\subsubsection*{Influenccia de la saturaci\'on en $V_{in_{max}}$}
La tensi\'on pico a pico m\'axima de salida del amplificador operacional es llamada 
tensi\'on de saturasi\'on $V_{sat}$. Te\'oricamente, este valor es igual a $V_{CC}$. Dado que $V_{out} = \rvert H(s) \rvert V_{in}$:

\begin{equation}
	V_{in_{max}} = \frac{V_{out_{max}}}{\rvert H(s) \rvert} = \frac{V_{sat}}{\rvert H(s) \rvert} = \frac{V_{CC}}{\rvert H(s) \rvert}
\end{equation}

Dado que en nuestro caso usamos $V_{CC} = \pm15V$, la expresi\'on que se obtiene es:

\begin{equation}
	V_{in_{max}} = \frac{15V}{\rvert H(s) \rvert} 
\end{equation}

%%%%%%%
\subsubsection*{- Configuraci\'on inversora}

\begin{table}[h!]
	\centering
	\begin{tabular}{c c c }%
		\bfseries  & $\bm{V_{in_{max}}}$ \\ \hline
		caso 1 & $5.79 \cdot 10^{-13} \sqrt{1.87 \cdot 10^{15} f^{2} + 8.68 \cdot 10^{24}}$\\
		caso 2 & $3.18 \cdot 10^{-12} \sqrt{6.17 \cdot 10^{13} f^{2} + 8.68 \cdot 10^{24}}$ \\
		caso 3 & $5.79 \cdot 10^{-13} \sqrt{1.87 \cdot 10^{15} f^{2} + 8.68 \cdot 10^{26}}$\\
		\hline
	\end{tabular}
	\caption{Circuito inversor: Tensi\'on de entrada m\'axima, limitada por saturaci\'on.}
	\label{vin_maxS}
\end{table}


\begin{figure}[H] %!ht
	\centering
	\includegraphics[width=10cm,height=10cm,keepaspectratio]{../EJ1/00GRAFICOS/teoricos/c1sat.png}
	\caption{Circuito inversor: Tensi\'on m\'axima de entrada limitada por saturaci\'on.}
	\label{c1sat}
\end{figure}

%%%%%%%%%%%
\subsubsection*{- Configuraci\'on no inversora}

\begin{table}[h!]
	\centering
	\begin{tabular}{c c c }%
		\bfseries  & $\bm{V_{in_{max}}}$ \\ \hline
		caso 1 & $ 5.79 \cdot 10^{-13} \sqrt{1.87 \cdot 10^{15} f^{2} + 8.68 \cdot 10^{24}}$\\ \\
		caso 2 & $3.18 \cdot 10^{-12} \sqrt{6.17 \cdot 10^{13} f^{2} + 8.68 \cdot 10^{24}}$ \\ \\
		caso 3 & $5.79 \cdot 10^{-13} \sqrt{1.87 \cdot 10^{15} f^{2} + 8.68 \cdot 10^{26}} $\\
		\hline
	\end{tabular}
	\caption{Circuito no inversor: Tensi\'on de entrada m\'axima, limitada por saturaci\'on.}
	\label{vin_maxS}
\end{table}

\begin{figure}[H] %!ht
	\centering
	\includegraphics[width=10cm,height=10cm,keepaspectratio]{../EJ1/00GRAFICOS/teoricos/c2sat.png}
	\caption{Circuito no inversor: Tensi\'on m\'axima de entrada limitada por saturaci\'on.}
	\label{c2sat}
\end{figure}

%%%%%%%%%%%%%%%%%%%%%%%%%%%%
\subsubsection*{Combinaci\'on del efecto de slew rate y saturaci\'on sobre la tensi\'on m\'axima de entrada}

\begin{figure}[H] %!ht
	\centering
	\includegraphics[width=10cm,height=10cm,keepaspectratio]{../EJ1/00GRAFICOS/teoricos/c1vimax.png}
	\caption{Circuito inversor: Tensi\'on m\'axima de entrada limitada por slew rate y saturaci\'on.}
	\label{c1vinmax}
\end{figure}

\begin{figure}[H] %!ht
	\centering
	\includegraphics[width=10cm,height=10cm,keepaspectratio]{../EJ1/00GRAFICOS/teoricos/c2vimax.png}
	\caption{Circuito no inversor: Tensi\'on m\'axima de entrada limitada por slew rate y saturaci\'on.}
	\label{c2vinmax}
\end{figure}


\subsubsection{Mediciones y resultados obtenidos}
Para encontrar de forma pr\'actica la tensi\'on de entrada m\'axima, se utiliz\'o el osciloscopio en 
modo MATH, FFT, para observar la tensi\'on de entrada a partir de la cu\'al comenzaban a aparecer m\'as arm\'onicos
de lo observado a tensi\'on baja, lo cual indica que se comienza a distorsionar la se\~nal de salida. se fue subiendo la tensi\'on pico a pico de una senoidal de entrada para ver
a partir de qu\'e valor ocurr\'ia dicha distorsi\'on. Utilizando una senoidal de 100KHz, para el circuito inversor se obtuvo:

\begin{table}[h!]
	\centering
	\begin{tabular}{c c c}%
		\bfseries  & $V_{in} m\'axima$ \\ \hline
		caso 1 & $415mV$\\
		caso 2 & $270mV$ \\
		caso 3 & $285mV$ \\
		\hline
	\end{tabular}
	\caption{Tensi\'on de entrada m\'axima sin distorsi\'on a la salida.}
	\label{casos}
\end{table}


\subsubsection{Presencia de la resistencia $R_4$}
Se nos pidi\'o analizar la influencia de la resistencia $R_4$ en el circuito de configuraci\'on inversora.
En las expresiones de $V_{out}/V_{in}$ y $Z_{in}$ obtenidas para este circuito se observa que
esta resistencia no influye. Por lo tanto, variando dicho valor, la respuesta en frecuencia y la impedancia
de entrada al circuito seguir\'a siendo la misma.

\subsubsection{Ausencia de la resistencia $R_3$}

Para el caso del circuito no inversor, si $R_3$ vale cero, la impedancia de entrada valdr\'a lo mismo 
que $R_3$ en el c\'alculo te\'orico. No ocurre lo mismo para el circuito inversor. En este \'ultimo, 
la impedancia de entrada no depende de $R_3$, por lo que el valor de dicha impedancia no cambiar\'a
al anular $R_3$.

Al observar lo que ocurre con $V_{out}/V_{in}$, se puede ver que anulando $R_3$, disminuyela ganancia del circuito inversor, mientras que aumenta en el caso del no inversor.

\subsubsection{Selecci\'on de amplificador operacional para altas frecuencias}
Según el análisis realizado no sería conveniente emplear el lm324 para una se\~nal cuya 
frecuencia oscile entre 0.3Mhz y 2Mhz. En este punto se puede observar que la ganancia 
en tensión es casi nula, por lo que el amplificador operacional dificilmente podrá ser usado 
para amplificar. Esto se puede observar en el siguiente gráfico, extraído de la wwb de Texas 
Instruments.

\begin{figure}[H] %!ht
	\centering
	\includegraphics[width=10cm,height=10cm,keepaspectratio]{../EJ1/00GRAFICOS/imagenes/EJ1_avol_max477.png}
	\caption{Ganancia a lazo abiero del LM324 en funci\'on de la frecuencia.}
	\label{i1}
\end{figure}

Por otro lado también se debe tener en cuenta el efecto del slew rate. Con una tensión pico a pico de 1V y variando la frecuencia en el rango mencionado arriba y además suponiendo ganancia unitaria tendríamos un coeficiente de slew rate que oscila entre 1.8 y 12.56 volts por microsegundo. Esto supera ampliamente al propio coeficiente máximo extraído de la datasheet del componente. Este úlltimo vale 0.5 volt por microsegundo. Consecuentemente, la se\~nal de salida se verá gravemente distorsionada a causa de este efecto, aún en el mejor de los casos de amplificación.

Por otro lado este amplificador operacional tiene un coeficiente de slew rate de 1100 volts por microsegundo, cuatro veces superior en orden de magnitud al del LM324. Cabe destacar que este amplificador está pensado para frecuencias muy superiores al rango solicitado, por lo que su aplicación en dicho rango seguramente será poco eficiente en términos de costo  beneficio.

Otra opción un tanto más lógica en terminos de costos sería la serie OPA141 de Texas Instruments. Estos amplificadores están dise\~nados para operar en frecuencias del orden de los 10MHz. En cuanto al slew rate, el coeficiente característico de esta familia es de 20 volt por microsgundo (lo que implica que las se\~nales mencionadas en el apartado anterior entran en el rango de no distorsión). Respecto a la ganancia  del operacional, en el siguiente gráfico se puede veirificar que la ganancia a lazo cerrado es superior a la unitaria, por lo que puede ser configurable para que valga uno.

\begin{figure}[H] %!ht
	\centering
	\includegraphics[width=10cm,height=10cm,keepaspectratio]{../EJ1/00GRAFICOS/imagenes/EJ1_avcl_OPA141.png}
	\caption{Ganancia a lazo cerrado del OPA141}
	\label{i2}
\end{figure}

Por último, psteodemos mencionar al modelo TSX920 de ST. Al igual que el anterior este circuito integrado está dise\~nado para trabajar a frecuencias del orden de los 10MHz. En este caso el slew rate está afectado por el signo de la pendiente de la se\~nal, valiendo su coeficiente 17.7 y 19.6 volt por microsegundo con pendiente positiva y negativa respectivamente. Desde este punto de vista es apto para la aplicación solicitada.  Por otro lado, abajo se puede observar que la ganancia a lazo abierto es suficiente para obtener gananicia unitaria, en las frecuencias de trabajo del amplificador.

\begin{figure}[H] %!ht
	\centering
	\includegraphics[width=10cm,height=10cm,keepaspectratio]{../EJ1/00GRAFICOS/imagenes/EJ1_avol_tsx920.png}
	\caption{Ganancia a lazo cerrado del TXS920.}
	\label{i2}
\end{figure}




















