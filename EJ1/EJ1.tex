
\section*{Ejercicio 1: Comportamiento de amplificadores operacionales}
En este ejercicio se analizan distintas caracter\'isticas de circuitos con amplificadores operacionales. Primero se utiliza un circuito con configuraci\'on inversora y luego otro con configuraci\'on no inversora.

\subsection{Configuraci\'on inversora}

\begin{figure}[h!]
 \begin{center}
    \begin{circuitikz}
\ctikzset{bipoles/length=1cm}
\draw
(0, 0) node[op amp] (opamp) {}
(opamp.-) to[short] (-2,0.35)
to[R,l_=$R_1$,-o] (-4, 0.35) node[anchor=east]{IN}
(-4,-1.5) to[sV,v=$V_{in}$] (-4,0.35)
(-4,-1.5) node[ground]{}

(-2,0.35) to[R=$R_3$] (-2,-1.5) node[ground]{}

(opamp.-) to[short,*-] ++(0,0.5) coordinate (leftC)
to[R=$R_2$] (leftC -| opamp.out)
to[short,-*] (opamp.out)  node[anchor=west]{OUT}

(opamp.+) -- (-1,-0.35) to (-1,-1.5) node[ground]{}
(opamp.out) to[short] (0.85,-0.4)
%to[R=$R_4$] (0.85,-1)  to[short] (0.85,-1.5)node[ground]{}
 to[R=$R_4$] (0.85,-1.5) node[ground]{}
;
    \end{circuitikz}
    \caption{Configuraci\'on inversora}
\end{center}
\end{figure}


\subsection{Configuraci\'on no inversora}

\begin{figure}[h!]
 \begin{center}
    \begin{circuitikz}
\ctikzset{bipoles/length=1cm}
\draw
(0, 0) node[op amp] (opamp) {}
%(opamp.-) to[short] (-2,0.35)
%to[R,l_=$R_1$,-o] (-4, 0.35) 
%(-4,0.35) to[short] (-4,-1.5) node[ground]{}
(opamp.-) ++(0,0.5) coordinate (leftC) to[R,l_=$R_1$] (-3,0.8) node[ground]{}

(opamp.-) to[short,*-] ++(0,0.5) coordinate (leftC)
to[R=$R_2$] (leftC -| opamp.out)
to[short,-*] (opamp.out)  node[anchor=west]{OUT}

(opamp.+)  to[short] (-1,-0.35)
to[R=$R_3$] (-3,-0.35) 
(-1,-0.35) to[R=$R_4$] (-1,-2) node[ground]{}

(-3,-2) to[sV,v=$V_{in}$] (-3,-0.35) node[anchor=east]{IN}
(-3,-2) node[ground]{}
;
    \end{circuitikz}
    \caption{Configuraci\'on inversora}
\end{center}
\end{figure}

%%%%%%%%%%%%%%%%%%%%%%%%%%%%%%%%%%
Circuito 1 Vo/Vi CASO 1
\begin{equation}
- \frac{7000000000000.0}{2088908.62808113 s + 700131250000.0}
\end{equation}

Circuito 1 Vo/Vi CASO 2
\begin{equation}
- \frac{700000000000.0}{298415.518297304 s + 700018750000.0}
\end{equation}

Circuito 1 Vo/Vi CASO3
\begin{equation}
- \frac{7000000000000.0}{11936620.7318921 s + 70000750000000.0}
\end{equation}
%%%%%%%%%%%%%%%%%%%%%%%%%%%%%%%%%

CIRCUITO 1 ZIN
caso1
\begin{equation}
\frac{437.676093502712 s + 280027500.0}{0.0159154943091895 s + 112001.0}
\end{equation}
CON PUNTA:
\begin{equation}
\frac{1.0 \left(3.64730077918927 \cdot 10^{20} s + 2.3335625 \cdot 10^{26}\right)}{4376760935.02712 s^{2} + 1.60996599321165 \cdot 10^{16} s + 9.33575022916667 \cdot 10^{22}}
\end{equation}

inverter: Zin caso2=
\begin{equation}
\frac{79.5774715459477 s + 280005000.0}{0.0159154943091895 s + 112001.0}
\end{equation}
CON PUNTA:
\begin{equation}
\frac{1.0 \left(6.63145596216231 \cdot 10^{19} s + 2.333375 \cdot 10^{26}\right)}{795774715.459477 s^{2} + 1.60695933802868 \cdot 10^{16} s + 9.33575004166667 \cdot 10^{22}}
\end{equation}

inverter: Zin caso3=
\begin{equation}
\frac{437.676093502712 s + 2800027500.0}{0.0159154943091895 s + 112001.0}
\end{equation}
CON PUNTA:
\begin{equation}
\frac{1.0 \left(3.64730077918927 \cdot 10^{20} s + 2.33335625 \cdot 10^{27}\right)}{4376760935.02712 s^{2} + 4.12996599321165 \cdot 10^{16} s + 9.35675022916667 \cdot 10^{22}}
\end{equation}
%%%%%%%%%%%%%%%%%%%%%%%%%%%%%%%%%%%%%%%
CIRCUITO 2
ZIN
caso1
\begin{equation}
Zin = R3 + R4 = 12.5kohm
\end{equation}
CON PUNTA:
\begin{equation}
\frac{1.0 \left(1.34287121997052 \cdot 10^{38} s - 8.4366609366572 \cdot 10^{43}\right)}{1.61144546396462 \cdot 10^{27} s^{2} + 9.74400011449559 \cdot 10^{33} s - 6.75776541020242 \cdot 10^{39}}
\end{equation}

NONinverter: Zin caso2=
\begin{equation}
12.5kohm
\end{equation}
CON PUNTA:
\begin{equation}
\frac{1.0 \left(3.35718845716807 \cdot 10^{36} s - 2.10935402480826 \cdot 10^{43}\right)}{4.02862614860169 \cdot 10^{25} s^{2} + 1.57883363155109 \cdot 10^{31} s - 1.68959257386991 \cdot 10^{39}}
\end{equation}

NONinverter: Zin caso3=
\begin{equation}
125kohm
\end{equation}
CON PUNTA:
\begin{equation}
\frac{1.0 \left(1.34287632288 \cdot 10^{39} s - 8.43749203068526 \cdot 10^{46}\right)}{1.611451587456 \cdot 10^{28} s^{2} - 1.00162173591756 \cdot 10^{36} s - 6.83436854484906 \cdot 10^{41}}
\end{equation}

%%%%%%%%%%%%%%%%%%%%%%%%%%%%%%%%%%%%%
%%%%%%%%%%%%%%%%%%%%%%%%%%%%%%%%%%%
%%%%%%%%%%%%%%%%%%%%%%%%%%%%%%%%%%%
zin circuito1 caso1 teorica:
\begin{equation}
\frac{1.30885711543124 \cdot 10^{16} s - 3.6842622243421 \cdot 10^{27}}{2748284324476.07 s - 7.73606889861856 \cdot 10^{23}}
\end{equation}

zin circuito1 caso2 teorico:
\begin{equation}
\frac{2500.0 \left(1202441.0 s - 5.38729407038047 \cdot 10^{15}\right)}{802241.0 s - 3.59439807358207 \cdot 10^{15}}
\end{equation}

zin circ1 caso2 teo BIEN:
\begin{equation}
\frac{1.868890907484 \cdot 10^{15} s - 5.26102936560593 \cdot 10^{25}}{498752424613.223 s - 1.404061747493 \cdot 10^{22}}
\end{equation}

zin circ1 caso3 teo:
\begin{equation}
\frac{7.49205761516139 \cdot 10^{17} s - 2.10926458373408 \cdot 10^{28}}{27480414888480.5 s - 7.73677133618684 \cdot 10^{23}}
\end{equation}


zin circ2 caso1 teo:
\begin{equation}
\frac{1.61144546396462 \cdot 10^{20} s - 1.01239931239886 \cdot 10^{26}}{1.28915648576337 \cdot 10^{16} s - 8.09919449911891 \cdot 10^{21}}
\end{equation}

zin circ2 caso2 teo:
\begin{equation}
\frac{4.02862614860169 \cdot 10^{18} s - 2.53122482976991 \cdot 10^{25}}{322290120536142.0 s - 2.02497986381413 \cdot 10^{21}}
\end{equation}


zin circ2 caso3 teo:
\begin{equation}
\frac{1.611451587456 \cdot 10^{21} s - 1.01249904368223 \cdot 10^{29}}{1.28916241588535 \cdot 10^{16} s - 8.09999234945065 \cdot 10^{23}}
\end{equation}

