
\section*{Ejercicio 1: Comportamiento de amplificadores operacionales}
En este ejercicio se analizan distintas caracter\'isticas de circuitos con amplificadores operacionales. Primero se utiliza un circuito con configuraci\'on inversora y luego otro con configuraci\'on no inversora.

\subsection{Configuraci\'on inversora}

\begin{figure}[h!]
 \begin{center}
    \begin{circuitikz}
\ctikzset{bipoles/length=1cm}
\draw
(0, 0) node[op amp] (opamp) {}
(opamp.-) to[short] (-2,0.35)
to[R,l_=$R_1$,-o] (-4, 0.35) node[anchor=east]{IN}
(-4,-1.5) to[sV,v=$V_{in}$] (-4,0.35)
(-4,-1.5) node[ground]{}

(-2,0.35) to[R=$R_3$] (-2,-1.5) node[ground]{}

(opamp.-) to[short,*-] ++(0,0.5) coordinate (leftC)
to[R=$R_2$] (leftC -| opamp.out)
to[short,-*] (opamp.out)  node[anchor=west]{OUT}

(opamp.+) -- (-1,-0.35) to (-1,-1.5) node[ground]{}
(opamp.out) to[short] (0.85,-0.4)
%to[R=$R_4$] (0.85,-1)  to[short] (0.85,-1.5)node[ground]{}
 to[R=$R_4$] (0.85,-1.5) node[ground]{}
;
    \end{circuitikz}
    \caption{Configuraci\'on inversora}
\end{center}
\end{figure}


\subsection{Configuraci\'on no inversora}

\begin{figure}[h!]
 \begin{center}
    \begin{circuitikz}
\ctikzset{bipoles/length=1cm}
\draw
(0, 0) node[op amp] (opamp) {}
%(opamp.-) to[short] (-2,0.35)
%to[R,l_=$R_1$,-o] (-4, 0.35) 
%(-4,0.35) to[short] (-4,-1.5) node[ground]{}
(opamp.-) ++(0,0.5) coordinate (leftC) to[R,l_=$R_1$] (-3,0.8) node[ground]{}

(opamp.-) to[short,*-] ++(0,0.5) coordinate (leftC)
to[R=$R_2$] (leftC -| opamp.out)
to[short,-*] (opamp.out)  node[anchor=west]{OUT}

(opamp.+)  to[short] (-1,-0.35)
to[R=$R_3$] (-3,-0.35) 
(-1,-0.35) to[R=$R_4$] (-1,-2) node[ground]{}

(-3,-2) to[sV,v=$V_{in}$] (-3,-0.35) node[anchor=east]{IN}
(-3,-2) node[ground]{}
;
    \end{circuitikz}
    \caption{Configuraci\'on inversora}
\end{center}
\end{figure}

zin circuito1 caso1 teorica:
\begin{equation}
\frac{2500.0 \left(21052907.0 s - 9.43171129431577 \cdot 10^{17}\right)}{11051507.0 s - 4.95108409511588 \cdot 10^{17}}
\end{equation}






