
\section*{Ejercicio 6}

\subsection{Introducción}

En este ejercicio se realizará el diseño de un circuito que adapta una señal de tensión proveniente de un sensor de temperatura LM324. 
Luego, se implementará el mismo en un PCB y se analizarán los resultados obtenidos.

\subsection{Diseño del circuito}

Como primer paso se deben definir las entradas y salidas del sistema para poder comenzar con su diseño. Respecto a la entrada, la fuente de la misma es el sensor de temperatura LM324.
 Según se pudo consultar en su datasheet, la ganancia del mismo es de $100 \frac{mV}{°C}$.Por otro lado, el rango de temperaturas especificado de funcionamiento del circuito va desde $35°C$ hasta $45°C$.
 Luego, el rango de tensiones de entrada es de $350mV$ a $450mV$, lineal respecto de la temperatura. El rango de salida del circuito está comprendido entre $0V$ y $5V$, a  $35°C$ y $45°C$ respectivamente. 

 
 Como se puede observar, tanto la entrada como la salida del circuito son rangos lineales, por lo que la adaptación implica solamemte un escalamiento y un corrimiento aplicados sobre la señal de entrada.
  Teniendo en cuenta los circuitos típicos observados en clase se decidió emplear un sumador inversor conectado en cascada a un amplificador inversor, como se puede observar en la figura~\ref{fig:EJ6_circuito}.  
  
\begin{figure}[H]
    \centering
    \includegraphics[width=0.9\textwidth]{dummypath/EJ6_circuito.png}
    \caption{Circuito adaptador}
    \label{fig:EJ6_circuito} 
\end{figure}

Dado que se utilizan dos amplificadores operacionales se creyó conveniente emplear el integrado TL082. La ecuación que caracteriza a dicho circuito se reproduce a continuación.

\begin{equation}[H]
    Acá va la ecuación del sistema.
    \label{fig:EJ6_ecuacion_sistema} 
\end{equation}

Si suponemos que el amplificador inversor tiene ganancia unitaria (esto es, $Rx=Rx$) obtenemos

\begin{equation}[H]
    Acá va la ecuación del sistema simplificada.
    \label{fig:EJ6_ecuacion_sistema_simplificada} 
\end{equation}

En la ecuación anterior se pueden apreciar las dos etapas de adaptación mencionadas anteriormente (escalamiento y corrimiento). Ahora si operamos sobre esta simplificación obtenemos

\begin{equation}[H]
    Acá va la ecuación del sistema simplificada final.
    \label{fig:EJ6_ecuacion_sistema_simplificada_final} 
\end{equation}

